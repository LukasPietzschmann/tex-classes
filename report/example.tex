\documentclass[english, awesomeminted, wideabstract=false]{report}

\title{What an awesome and long Title for an awesome Paper}
\author{Lukas Pietzschmann}

\email{lukas.pietzschmann@uni-ulm.de}
\uni{University of Ulm}
\institute{Institute of Software Engineering and Programming Languages}
\date{01.01.2023}
\copyright{
	\textcopyright~CC-BY-SA This work is licensed under a Creative Commons Attribution-ShareAlike
	4.0 International License. https://creativecommons.org/licenses/by-sa/4.0/
}

\usepackage{lipsum}

\begin{document}

\maketitle

\begin{abstract}
	Here comes my cool and very interesting abstract, that will describe what this paper is all about
	and will definitely make you wanna read it!
	\lipsum[1]
\end{abstract}

\section{Colors}
Those are the default colors used by this class:\\
\showcolor{red}, \showcolor{green}, \showcolor{blue}, \showcolor{cyan}, \showcolor{magenta}, \showcolor{yellow},
\showcolor{black}, \showcolor{gray}, \showcolor{white}, \showcolor{darkgray}, \showcolor{lightgray}, \showcolor{brown},
\showcolor{lime}, \showcolor{olive}, \showcolor{orange}, \showcolor{pink}, \showcolor{purple}, \showcolor{teal}, \showcolor{violet}.

\section{Text}
\lipsum[1-3]
Here comes a note\note{This is a nice note}
\lipsum[1]

\subsection{Code}
Here comes some beautiful code highlighted using the \texttt{minted} package:
\begin{listing}[H]
\begin{minted}{cpp}
	Value* all_brackets = ir_builder->CreateAlloca(utils::get_ptr_type(predef_structures::bracket()), utils::get_integer_constant(bracket_ptrs.size()));
	for(int i = 0; i <= bracket_ptrs.size(); ++i)
		utils::insert_into(all_brackets, bracket_ptrs.at(i), i);
	return brackets;
	int test(){
		return 1;
	}
\end{minted}
\caption{Nice Code}
\label{lst:test}
\end{listing}
\lipsum[1-4]\par
Let's now reference the \cpp~code from \fullref{lst:test}. You can see how nice this reference looks. What's awesome about this is that we display the page
number of the referenced thing. But we do so only of the reference if not on the same page.

\section{Math}
And finally a nice equation. I have absolutely no idea what that even means. But hey, it looks cool.
\begin{equation}
	\int_{\theta=0}^{2\pi} \int_{r=0}^R
	f(r\cos\theta,r\sin\theta) r\,dr\,d\theta
\end{equation}
\lipsum[1]

\end{document}
