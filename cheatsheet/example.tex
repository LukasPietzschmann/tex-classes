\documentclass[columns=3]{cheatsheet}

\title{Python-Cheatsheet}
\author{Lukas Pietzschmann}
\email{lukas.pietzschmann@uni-ulm.de}
\date{12.01.2022}


\begin{document}

\maketitle

\section{Abstract}
Dieses kurze Cheat-Sheet fasst die wichtigsten Konzepte von Python3 mithilfe
prägnanter Beispiele zusammen. Vor allem werden hier allerdings für die Übung relevante Dinge vorgestellt.

\section{Variablen}
Im Gegensatz zu vielen anderen Programmiersprachen müssen in Python Variablen nicht explizit deklariert
werden. Variablen sind ab ihrer ersten Initialisierung verwendbar.\par
Typen müssen dabei nie explizit angegeben werden! Folgendes Beispiel zeigt einige relevante Typen:
\begin{lstlisting}[language=python]
s = "ich bin ein String"
i = 42       #int
f = 3.1415   #float
\end{lstlisting}

\section{Funktionen}
Funktionen sind wie auch Variablen ungetypt. Sie können mit dem Schlüsselwort \lstinline[language=python]{def}
erstellt werden.
\begin{lstlisting}[language=python]
def fib(n):
	if n == 1 or n == 0:
		return 0
	else
		return fib(n-1) + fib(n-2)

fib(100) # <!\bot!>
\end{lstlisting}

\section{Anonyme Funktionen (lambdas)}
Wie viele modernen Sprachen unterstützt auch Python Konzepte der funktionalen Programmierung. Lambdas sind eines davon.\par
Eine (anonyme) Funktion kann dabei mit dem Schlüsselwort \lstinline[language=python]{lambda} erzeugt werden.
\begin{lstlisting}[language=python]
fun = lambda a, b: a + b
fun(1, 2) # 3
\end{lstlisting}
Da Lambdas ausschließlich Ausrücke und keine Statements enthalten dürfen und dadurch recht eingeschränkt sind, werden sie häufig
dazu verwendet simple Funktionen an andere Funktionen als Argument zu übergeben.
\begin{lstlisting}[language=python]
map(lambda elem: print(elem), [1, 2, 3])
# 1 2 3
\end{lstlisting}


\section{Listen}
Listen sind in Python nicht in ihrer Größe beschränkt und können Daten mit unterschiedlichen Typen enthalten.
\begin{lstlisting}[language=python]
l = [1, 2]
l.append(3)
print(l)    # [1, 2, 3]
l.pop()
print(l)    # [1, 2]
\end{lstlisting}

\section{Tupel}
Tupel sind recht ähnlich zu Listen, mit dem Unterschied, dass sie nach ihrer Erstellung nicht mehr verändert
werden können.
\begin{lstlisting}[language=python]
t = (1, 2, 3)
\end{lstlisting}

\section{Operationen auf Listen und Tupeln}
Tupel und Listen (und viele anderen Datenstrukturen) teilen sich einige Operationen die auf ihnen zulässig sind.
Hier eine kleine Auswahl der wichtigsten:

\begin{description}
	\item[Element-Zugriff] Mit dem Klammer-Operator \texttt{[]} können Elemente anhand ihres Indexes ausgewählt werden.
		Dabei wird ab 0 indiziert.
		\begin{lstlisting}[language=python]
l = [1, 2, 3]
print(l[0]) # 1
		\end{lstlisting}

	\item[Slicing] Soll nicht nur ein Element, sondern gleich mehrere abgefragt werden, kann ebenfalls der Klammer-Operator
		mit einem Slice verwendet werden. Die Syntax sieht folgendermaßen aus: \texttt{[<start>:<ende>:<schrittweite>]}.
		Dabei ist jeder Parameter Optional
		\begin{lstlisting}[language=python]
l = [1, 2, 3]
print(l[1:]) # [2, 3]
print(l[:1]) # [1]
print(l[::2]) # [1, 3]
		\end{lstlisting}

	\item[Iteration] Soll eine Operation für jedes Element einer Liste oder eines Tupels ausgeführt werden, kann eine
		\lstinline[language=python]{for}-Schleife verwendet werden.
		\begin{lstlisting}[language=python]
l = [1, 2, 3]
for i in l:
	print(i) # 1 2 3
		\end{lstlisting}
\end{description}

\section{Imports}
Um andere Dateien zu importieren kann das Schlüsselwort \lstinline[language=python]{import} verwendet werden.
\begin{lstlisting}[language=python]
import math
print(math.pi) # 3.14...
\end{lstlisting}
Sollen nicht alle Symbole der Datei, sondern nur eine ausgewählte Menge importiert werden, kann zusätzlich
\lstinline[language=python]{from} verwendet werden.
\begin{lstlisting}[language=python]
from math import pi
print(pi) # 3.14...
\end{lstlisting}

\section{Diverse Details}
\begin{description}
	\item[Code-Struktur] Python geht den ungewöhnlichen Weg und kennzeichnet Blöcke an Code nicht durch geschweifte
		Klammern, sondern durch Einrückung. Die Anzahl der Tabs (oder Leerzeichen) ist dabei erstmal egal, sollte
		aber konsistent bleiben.
	\item[Main-Funktion] Python besitzt keine echte Main-Funktion. Code wird einfach von der ersten Zeile ab gelesen
		und ausgeführt. Folgender Trick kann allerdings eine Main-Funktion halbwegs gut emulieren:
		\begin{lstlisting}[language=python]
def main():
	print("Hallo Welt")

if __name__=="__main__":
	main()
		\end{lstlisting}
	\item[Topkonvertierung] In der Übung wird es einige Stellen geben an denen Strings zu Zahlen umgewandelt werden sollen.
		Hier das Vorgehen:
		\begin{lstlisting}[language=python]
s = "420"
i = int(s)
		\end{lstlisting}
		Das Umwandeln von Typen funktioniert in der Regel immer über deren Konstruktoren.

\section{Hilfreiche Funktionen und Methoden}
Hier eine kleine Auswahl nützlicher Funktionen und Methoden:
\begin{description}
	\item[print] Die \lstinline[language=python]{print}-Funktion gibt die übergebenen Argumente als Text auf der Standardausgabe aus.
	\item[split] Diese Methode kann auf Strings aufgerufen werden und teilt den String an dem übergebenen Trenner.
		\begin{lstlisting}[language=python]
s = "Hallo Welt!"
elems = s.split(" ")
for e in elems:
	print(e) # "Hallo" "Welt"
		\end{lstlisting}
	\item[upper, lower] Diese Methode (ebenfalls aus der String-Klasse) geben einen neuen String zurück, der nur aus Groß-, beziehungsweise
		Kleinbuchstaben besteht.
		\begin{lstlisting}[language=python]
s = "LoL"
print(s.lower()) # "lol"
print(s.upper()) # "LOL"
		\end{lstlisting}
	\item[join] \texttt{join} ist wieder eine Methode die auf Strings angewendet werden kann. Die bekommt eine Liste an beliebigen Elementen
		übergeben und konkateniert diese mit dem String als Trenner.
		\begin{lstlisting}[language=python]
print(";".join([1, 2, 3])) # "1;2;3"
		\end{lstlisting}
	\item[range] Diese Funktion kann dabei helfen eine \enquote{\textit{traditionelle}} For-Schleife zu emulieren.
		Sie kann wie folgt verwendet werden:
		\begin{lstlisting}[language=python]
for i in range(1, 3):
	print(i) # 1 2 3
		\end{lstlisting}
\end{description}



\end{description}


\end{document}
